% !TeX encoding = UTF-8
% !TeX spellcheck = de_DE
%Dokumentenklasse: Papierformat, Schriftgröße, Doppelseitig
\documentclass[12pt,a4paper,twoside]{article}

%-----------------------------------------------------------------------------------------------------------------------------
%Bestimmt die Sprache der autom. eingefügten Worte, z. B. Inhaltsverzeichnis
%Letztgenannte Sprache ist aktiv
%Wechsel mit \selectlanguage{USenglish} im Dokument möglich
\usepackage[USenglish, ngerman]{babel}
%Einstellung der Randabstände
\usepackage[left={2.5cm},right={2.5cm},top={2cm},bottom={2.5cm}]{geometry}
%Schriftart
\usepackage{helvet}
%Textverschönerungen
\usepackage{lmodern}
\usepackage{microtype}
%Einbindung von Graphiken
\usepackage{graphicx}
%Bearbeitung von Kopf- und Fusszeile
\usepackage{fancyhdr}
%Aktivierung verschiedener Umgebungen, in denen der Mathematikmodus aktiv ist
\usepackage{mathtools}
\usepackage{amsthm}
\usepackage{amssymb}
%aktiviert Hyperlinks
\usepackage[pageanchor=false]{hyperref}
%Übersetzt die Tastatureingaben für LaTex
\usepackage[utf8]{inputenc} %Bei Umlautproblemen unter MacOS: \usepackage[applemac]{inputenc}
\usepackage[T1]{fontenc}
%Stellt das Eurozeichen € (\euro) zu Verfügung
\usepackage{eurosym}
%Textbox
\usepackage{fancybox}
%Farbe
\usepackage{xcolor}
%Todo-Markierungen hinzufügen
\usepackage{todonotes}
%\usepackage[disable]{todonotes}

%Zeilenabstand
\renewcommand{\baselinestretch}{1.24}
%Text nicht einrücken
\setlength{\parindent}{0pt}
%Fuß- und Kopfzeile
\fancyhf{}
\setlength{\headheight}{15pt}
\fancyhead[LE,RO]{\thepage}
\fancyhead[RE,LO]{\nouppercase{\rightmark}}
%Die ersten Seiten bekommen keine Kopf- oder Fußzeilen bis der "pagestyle" geändert wird
\pagestyle{empty}

%Befehle
\newtheorem{satz}{Satz}
\newtheorem{lemma}[satz]{Lemma}
\newtheorem{folgerung}[satz]{Folgerung}
\theoremstyle{definition}
\newtheorem{definition}[satz]{Definition}
\numberwithin{equation}{section}
\renewcommand{\proofname}{Beweis}
\definecolor{kitfarbe}{rgb}{0.004,0.588,0.51}
%Stellt nun auch die Eingabe "€" zur Verfügung
\DeclareUnicodeCharacter{20AC}{\euro}



\begin{document}
%-----------------------------------------------------------------------------------------------------------------------------
%Titelseite
\begin{titlepage}
  
\begin{center}
%	\includegraphics[scale=0.8]{kit_logo_de_farbe_positiv.jpg} 
	\includegraphics[scale=0.8]{KITlogo_4c_deutsch_RGB.pdf} 

\vspace{1cm}

	\Large{
	Fakultät für Wirtschaftswissenschaften \\
	Institut für Operations Research (IOR) \\
	Diskrete Optimierung und Logistik \\
	Prof. Dr. Stefan Nickel}  
     
\vspace{1cm}

	\Large Art der Arbeit % Seminararbeit, Bachelorarbeit oder Masterarbeit      
 
\vspace{1cm}
    
\setlength{\fboxrule}{3pt}

\begin{center}
\fcolorbox{kitfarbe}{white}{
	\parbox[c][3.5cm][c]{10cm}{ 
	\begin{center}
	\LARGE{Titel der Arbeit} \\
	\end{center}
	}}
\end{center}

\vspace{1cm}

	von \\
	Vorname Nachname \\
	Matr. Nr.: Matrikelnummer\\
	Studiengang \\

\vspace{1cm}

	Datum der Abgabe\\
	tt.mm.jjjj

\vspace{1cm}
	
	Betreuung: \\
	Titel Vorname Nachname 
    
\end{center}
  
\end{titlepage}

\cleardoublepage
%-----------------------------------------------------------------------------------------------------------------------------
%Eidesstattliche Erklärung
\vspace*{1cm}

{\Large \textbf{Eidesstattliche Erklärung}} 

\bigskip

Ich versichere hiermit wahrheitsgemäß, die Arbeit selbständig angefertigt, alle benutzten Hilfsmittel vollständig und genau angegeben und alles kenntlich gemacht zu haben, was aus Arbeiten anderer unverändert oder mit Abänderung entnommen wurde.\\
\vspace{1cm}

\textit{Datum} \hspace{8cm} \textit{Name}

\cleardoublepage
%-----------------------------------------------------------------------------------------------------------------------------
%Seitennummerierungen und Inhaltsverzeichnis
\rmfamily \pagestyle{fancy}
\renewcommand{\sectionmark}[1]{\markright{\thesection\ #1}}
\setcounter{secnumdepth}{4}
\pagenumbering{roman} \setcounter{page}{3} 
\tableofcontents
\newcounter{roemisch} \setcounter{roemisch}{\value{page}}
\clearpage
\mbox{}\thispagestyle{empty}\clearpage
\setcounter{page}{2} \pagenumbering{arabic}
%-----------------------------------------------------------------------------------------------------------------------------
%Beginn der Arbeit
\section{Einleitung}
Ein bisschen Einführung...
\vspace{\baselineskip}

Text... \\
Text... \\
Text...

\newpage

\section{Der Hauptteil}
Text...

\subsection{Erster Gliederungspunkt des Hauptteils}
Text...

\subsubsection{Ein weiterer Unterpunkt}
Text...

\subsubsection{Noch einer}
Text...

\subsubsection{Und noch einer}
Text...

\subsection{Bilder}
Ein Bild kann man auch in \LaTeX\ einfügen. Dabei bedeutet der Teil hinter dem \verb|\begin{figure}|, dass das Bild genau hier (h), oben (t), unten (b) oder auf einer speziellen Seite, welche nur Bilder und Tabellen enthält (p) eingefügt werden soll. Kombinationen sind hier ebenfalls zulässig, wie z.\,B. \lbrack htbp\rbrack.

\begin{figure}[h]
\begin{center}
\includegraphics[width=0.8\columnwidth]{KITlogo_4c_deutsch_RGB.pdf}
\caption{Das Uni-Logo als Beispielbild.} 
\label{bild_Referenz}
\end{center}
\end{figure}
Durch das \verb|\label| kann mit \verb|\ref| immer wieder
die Bildnummer referenziert werden. Das Unilogo ist z.\,B. Abbildung~\ref{bild_Referenz}.

\subsection{Tabellen}
Natürlich kann man auch Tabellen einfügen, wenn man will. Hier kommt z.\,B. Tabelle~\ref{tabelle_Referenz}.
\vspace{\baselineskip}

\begin{table}[h]
\centering
\begin{tabular}{|l||c|c|c|} \hline
& eine Spalte & noch eine & noch eine \\ \hline \hline
eine Zeile & 1 & 2 & 3 \\ \hline
weitere Zeile & 4 & 5 & 6 \\ \hline
\end{tabular}
\caption{Eine Beispieltabelle}
\label{tabelle_Referenz}
\end{table}

\subsection{Formeln}
Einfache Formeln schreibt man am besten so:
$$
a=\frac{b \cdot c}{d}
$$ 
oder im Text als $i=1/k$.
\vspace{\baselineskip}

Mehrzeilige Formeln kann man auch so schreiben
\begin{align} \label{formel_Referenz}
\delta(n,x) = &\ U_n(x) - U_n(x+1) \\
= &\ U_{n-1}(x) - U_{n-1}(x+1) + \left(U_n(x) - U_{n-1}(x)\right) - \left(U_n(x+1)-U_{n-1}(x+1)\right) \nonumber \\
= & \dots \nonumber
\end{align}
und als Formel (\ref{formel_Referenz}) referenzieren. 
\vspace{\baselineskip}

Ein lineares Programm:
\begin{align}
\begin{array}{llcll}
\text{min}  & \sum \limits_{j=1}^{k} y_{j}        &      &             & \\
\text{s.t.} & \sum \limits_{j=1}^{k} x_{ij}       & =    & 1           & \forall i \in \{1,...,n\} \\
            & \sum \limits_{i=1}^{n} x_{ij} g_{i} & \geq & y_{j} b_{j} & \forall j \in \{1,...,k\} \\
            & x_{ij}, y_{j}                       & \in  & \{0,1\}     & \forall i \in \{1,...,n\}, j \in \{1,...,k\} \\
\end{array}
\tag{LP}
\label{lin_Programm}
\end{align}

Eine stückweise definierte Funktion:

\begin{align}\label{formel2}
f(x) = \begin{cases}
0 & \text{if $x \leq 1$},\\
5 & \text{sonst}.
\end{cases}
\end{align}

\subsection{Sätze oder Beweise}
\begin{satz}
\label{satz_Referenz} 
Da kann man einen mathematischen Satz schreiben.
\end{satz}
\begin{proof}
Und beweisen...
\end{proof}
\begin{definition}
Oder eine Definition.
\end{definition}

\subsection{Quellen und Zitate}

Sinngemäße und wörtliche Wiedergabe fremden Gedankenguts muss gekennzeichnet werden. Wörtliche Zitate sind in unserem Fachbereich unüblich. Hier sind ein paar Beispiele

\begin{itemize}
	\item Dieses Kapitel basiert auf \cite{nickelor}.
	\item Das Problem ABC ist NP-schwer~\cite{melofl}.
	\item Das Problem XYZ lässt sich effizient lösen~\cite[Algorithmus 1]{melofl}.
\end{itemize}
Internetseiten~\cite{dash} sollten mit Abrufdatum zitiert werden.\\

Bei Abschlussarbeiten empfehlen wir die Benutzung eines verwaltungssystems wie Citavi oder Zotero. Weitere Infos dazu gibt es bei der KIT-Bibliothek (\url{https://www.bibliothek.kit.edu/cms/literaturverwaltung.php}).

\subsection{TODOs}

Dieser Satz muss umformuliert werden.\todo{noch nicht fertig}\\

\todo[inline]{Hier fehlt ein Beispiel.}

Alle TODOs lassen ist mit dem Befehl \verb|\listoftodos| anzeigen.

\subsection{Nützliche Links}

\begin{itemize}
	\item \LaTeX-Selbstlernkurs des Zentrums für Mediales Lernen (ZML)\\ 		
		\url{https://www.zml.kit.edu/4127.php}
	\item SchreibLABOR des House of Competence (HoC)\\
		\url{https://www.hoc.kit.edu/schreiblabor.php}
	\item E-Tutorials der KIT-Bibliothek, insbesondere Modul \grqq Schriftliche Ausarbeitung\glqq~ des Kurses \grqq Informationskompetenz\glqq \\
		\url{https://www.bibliothek.kit.edu/cms/e-tutorials.php}
\end{itemize}

\section{Zusammenfassung oder Fazit}
Am Ende der Arbeit sollte eine kurze Zusammenfassung stehen und
ein Ausblick auf weitere Aspekte, andere Herangehensweisen etc.
gegeben werden.

\newpage
%-----------------------------------------------------------------------------------------------------------------------------
%Anhang
\appendix
\section{Anhang}
\subsection{Erklärung}
Die Erklärung steht dann hier

\newpage
%-----------------------------------------------------------------------------------------------------------------------------
%Abbildungsverzeichnis
\listoffigures
\addcontentsline{toc}{section}{Abbildungsverzeichnis}
\newpage
%Tabellenverzeichnis
\listoftables
\addcontentsline{toc}{section}{Tabellenverzeichnis}
\newpage
%Literaturverzeichnis
\renewcommand\refname{Literaturverzeichnis}
\bibliography{Literatur_Vorlage} 
\bibliographystyle{plainurl}
\addcontentsline{toc}{section}{Literaturverzeichnis}

\end{document}